\section{Discussion and Future Work}
% 1. AI与人类类比推理的异同
% 概述:AI 可以通过模式匹配完成类比推理,但其对类比的 “理解” 是基于数据中统计到的相似性,而非人类式的深层认知或抽象思维。对 AI 而言,类比本质是 “特征相似性的迁移”,而人类的类比则是 “基于本质理解的创造性推理”。虽然逻辑上和产出的内容在很大程度上都会有相似性,但是本质的差异则是,人类知道自己在哪,要去何处。这种差异也反映了当前 AI 与人类智能在认知深度上的鸿沟。所以不应该轻易的给任何一种匹配模式下定义是创造性的或者非创造性的。

\textbf{Similarities and Differences between AI and Human Analogical Reasoning.}
Although AI can perform analogical reasoning through pattern matching\cite{gentner1983structure, gentner2011computational}, its understanding of analogy is rooted in statistically derived similarities within data rather than the uniquely human deep cognition or abstract reasoning\cite{webb2023emergent}. For AI, analogy is essentially a ``feature similarity transfer''— a process guided by algorithmic optimization of pattern alignment that identifies and replicates correspondences between datasets. In contrast, human analogy constitutes ``creative reasoning based on essential understanding''\cite{rangel2008framework, beaty2023associative, olsson2024analogies, weaver2010transformation}, where inter-domain connections are established through conscious abstraction, contextual interpretation, and the integration of tacit knowledge, thereby enabling the recognition of relational or functional similarities that go beyond mere feature overlap.  
% 尽管AI能够通过模式匹配执行类比推理,但其对类比的理解植根于数据中统计得出的相似性,而非人类特有的深度认知或抽象推理。对AI而言,类比本质上是一种“特征相似性迁移”——这一过程在模式对齐的算法优化引导下,识别并复制数据集之间的对应关系。相比之下,人类类比则是“基于本质理解的创造性推理”,其中领域间的关联通过有意识的抽象、情境解读和隐性知识的整合而建立,从而能够识别出超越单纯特征重叠的关系或功能相似性。

Humans can situate analogical reasoning within a broader framework of self-awareness\cite{moreno2016overcoming, chan2011benefits}, knowing their position in the problem space and the intended path of reasoning\cite{cemri2025multi}, a qualitative depth that AI reasoning lacks. This fundamental difference highlights a persistent cognitive gap: AI lacks the metacognitive grasp of purpose and context that drives human reasoning\cite{ball2019advancing}, reducing its analogical operations to mechanical pattern replication rather than conscious, meaning-constructive integration. 
% 人类能够将类比推理置于更广泛的自我意识框架中——清楚自己在问题空间中的位置以及推理的预期路径——这一点赋予了人类ai推理所不具备的质性深度。这一根本差异凸显了持久的认知鸿沟:AI缺乏驱动人类推理的、对目的和语境的元认知把握,使其类比操作沦为机械的模式复制,而非有意识的、意义建构性的整合。

Given this complexity, it is unreasonable to simplistically categorize these two processes as creative or non-creative. AI's pattern matching can produce novel outcomes through unexpected statistical associations, although human reasoning is rooted in a deeper understanding, it may sometimes be difficult to rapidly access due to excessive cognitive load\cite{richland2013reducing, lu2023differences}. These nuances demand a more refined evaluation, one that acknowledges both functional overlaps and cognitive gaps while rejecting binary judgments in favor of recognizing the distinct yet complementary roles they play in analogical thinking.
% 鉴于这种复杂性,将这两种过程简单归类为“创造性”或“非创造性”是不合理的。AI的模式匹配可通过意外的统计关联产生新颖结果,而人类的推理尽管植根于更深层的理解,有时也会因为认知负荷太大而难以快速调用。这些细微差别要求我们进行更细致的评估——既要承认两者的功能重叠与认知鸿沟,又要拒绝二元判断,转而认可它们在类比思维中所扮演的独特却互补的角色。


% 2. 通过技术设计平衡自动化与人的认知主导权
% 概述:DbA可以整合外在新知与激发内在经验,保护和激发创造性工作的主体感,对软件的控制感,对ai生成内容的信任感等,有了控制感才可以给人的创造力留出curation的余地,所以未来技术应当非常小心的维护设计师的认知主导权。
% 之前说code ai的出现使得每个人都可以成为全栈工程师,本文认为将创意过程拆解并且加入认知机制,流程化的使用设计工具可以让每个人都成为创业者/全链路的设计师,可以规划各种想要落地的产品,花更多时间去思考和挖掘自己自身的喜好和风格。激发个体的创造力,前人的经验只作为启发和结构化快速迁移操作,不给个人光芒蒙尘,并不规定某一集体价值观作为最终目的,而是希望每个使用者走自己喜欢的道路,但同时也不希望设计者带着不成熟的设计概念就进入市场造成亏空和资源浪费等等。



\textbf{Balancing Automation and Human Cognitive Dominance through Technological Design.}
Balancing this relationship entails neither the replacement of humans by technology nor human resistance to technology, but rather achieving a dynamic equilibrium between automated efficiency and cognitive agency through design\cite{hsueh2024counts, khanolkar2023mapping}, ultimately unlocking the deeper potential of creative practice. Specifically, technological design of DbA system should prioritize three core perceptions: (1) \underline{The sense of agency in creative work}, preserving the irreplaceable role of humans as initiators of ideas and final decision makers. (2) \underline{The sense of control over tool systems}, encompassing the comprehensibility of software logic, autonomy in function invocation, and the right to filter and modify AI-generated content; and (3) \underline{Trust in technological outputs}, a trust not rooted in blind faith in AI’s omniscience but in a clear understanding of its operational boundaries. At the core of these three perceptions lies the construction of control: When users can autonomously decide when to employ automation, how to adjust its outputs and why to reject certain suggestions, technology ceases to be a constraint on creativity and instead becomes scaffolding that reserves space for ``curation''\cite{rehm2017designing, frich2019mapping, chung2021intersection} in the design process, a space that serves as the very vessel of human uniqueness, allowing individuals to anchor their creative trajectory amidst the vast possibilities of AI rather than being swept away by algorithmic logic.  
% 平衡这种关系既非意味着技术对人的替代,也非人类对技术的抗拒,而是通过设计实现自动化效率与认知主导权的动态平衡,最终释放创造性实践的深层潜能。具体而言,技术设计需着力维护三重核心感知:1.创造性工作的主体感,即人作为创意发起者与最终决策者的不可替代性;2.对工具系统的控制感,包括对软件操作逻辑的可理解性、功能调用的自主性,以及对AI生成内容的筛选权与修正权;3.对技术输出的信任感,这种信任并非源于对AI全知的盲从,而是建立在对其运作边界的清晰认知之上。这三重感知的核心在于控制感的建构:当使用者能够自主决定何时启用自动化工具、如何调整其输出、为何拒绝某类建议时,技术便不再是束缚创造力的框架,而是为设计过程中的“curation”预留空间的脚手架——这种空间恰恰是人类独特性的载体,使人得以在AI提供的海量可能性中锚定自身的创意脉络,而非被算法逻辑裹挟。

The deeper significance of this balance lies in democratizing creative participation through technological empowerment. Decomposing creative processes into operable cognitive modules and embedding them in tool workflows is now driving a transformation: enabling more individuals to break through professional barriers and become ``full-stack designers'' or independent entrepreneurs\cite{bogenhold2014entrepreneurship}. This does not mean that technology simply reduces creative thresholds, but instead converts prior expertise into transferable structured tools, rather than rigid templates, unlocking individuals' end-to-end capabilities from conception to execution. For instance, designers no longer need to repeatedly learn foundational 3D modeling logic but can instead quickly invoke analogical rules between Augmented Manufacture structures and biological forms through tools\cite{yang2018recent}, focusing their energy on cultivating their own stylistic traits and value preferences\cite{schecter2025role}. In this process, external knowledge (e.g., industry data, cross-domain cases) serves as inspiration rather than directives, while internal experience (e.g., intuitive judgment, aesthetic inclinations) is amplified rather than suppressed by technological mechanisms, thereby preventing collective values\cite{schecter2025role} from homogenizing individual creativity\cite{anderson2024homogenization} and allowing each user to forge a unique path through autonomous exploration.  
% 这种平衡的深层意义,在于通过技术赋能实现创意参与的民主化。将创意过程拆解为可操作的认知模块,并嵌入工具流程的设计思路,正在推动一场变革:让更多个体突破专业壁垒,成为“全链路设计师”或自主创业者。这并非指技术简单降低创意门槛,而是通过将前人经验转化为可迁移的结构化工具,而非僵化的标准模板,释放个体从构想、规划到落地的全流程能力——例如,设计师无需重复学习基础建模逻辑,却能通过工具快速调用建筑结构与生物形态的类比规则,将精力集中于挖掘自身的风格特质与价值偏好。在此过程中,外在新知(如行业数据、跨域案例)作为启发而非指令存在,内在经验(如直觉判断、审美倾向)则被技术机制放大而非压制,从而避免集体价值观对个体创造力的同质化侵蚀,让每个使用者都能在自主探索中形成独特路径。

Yet this freedom must coexist with responsibility\cite{smits2019values, rigaud2022exploring}. While granting creative autonomy, technological design must embed prudent evaluation mechanisms, such as warning of potential risks through analogical analysis of historical cases\cite{thomas2013extending} or providing feasibility recommendations based on cross-domain performance data\cite{andriani2025perfume}, to prevent immature design concepts from entering the market in a hurry, causing resource waste and value erosion. This means the realization of new value depends not only on the stimulation of individual creativity but also on constructing a closed loop between free creation and rational constraint—where technology acts both as an amplifier of individual brilliance and a calibrator of creative practice, ultimately finding a dynamic equilibrium between preserving diversity and enhancing societal efficiency.
% 但这种自由需与责任共生。技术设计在赋予个体创意自主权的同时,必须嵌入审慎的评估机制:例如通过历史案例的类比分析预警潜在风险,或基于跨领域效能数据提供可行性建议,以避免不成熟的设计概念仓促进入市场导致的资源浪费与价值损耗。这意味着,新价值的实现不仅依赖于对个体创造力的激发,更在于构建自由创造与理性约束的闭环——技术既成为个体光芒的放大器,也成为创意实践的校准器,最终在保护多样性与提升社会效率之间找到动态平衡点。




\textbf{Future cross-spatiotemporal collaboration: multimodal knowledge sharing and accumulation.}
% 4. 未来的跨时空协作:多形式的知识共享与积累
% 1. 可穿戴设备/脑机接口刺激人脑产生某些类比概念的新形式的计算,从而脱离计算机界面辅助动手,写作,绘制,创作,各种内容。
% 2. 数据的隐私/模糊处理可能是知识共享的一大阻碍,这需要知识产权保护,也需要大型软件厂商对嵌入到软件的数据库进行更好地保护,有效的措施。也需要开源社区多做通用数据贡献,如有必要合成数据也是可以的。
At the technological level, emerging interaction modalities such as wearable devices and brain-computer interfaces (BCIs) can reshape the underlying logic of knowledge production and collaboration in future DbA, by directly linking human cognition with external environments\cite{jensen2011using}, they enable novel methods for stimulating\cite{birbaumer2006physiological}, transforming, and accumulating knowledge. For instance, BCIs can decode and stimulate neural activity patterns in real time\cite{jensen2011using}, facilitating the retrieval of knowledge and experience\cite{moore2010applications}; when creators sketch, wearables can detect their gestures and behaviors, interfacing with external knowledge bases to trigger cross-domain knowledge sharing\cite{birbaumer2006physiological, gao2021interface, nicolas2012brain}. Future interface-free collaboration could transform manual knowledge retrieval into contextually triggered exchanges, allowing fragmented knowledge across time and space to be dynamically reassembled through human actions—thereby offering new pathways for computing, sharing, and accumulating individual experience and collective intelligence through dynamic interaction.  
% 在技术层面,可穿戴设备、脑机接口等新兴交互形态及技术在未来的类比设计中的应用,可以重塑知识生产与协作的底层逻辑——通过直接连接人脑认知与外部环境,构建起刺激、转化、积累知识的新方法。如,脑机接口可通过捕捉神经活动模式,实时解析并刺激人脑的知识与经验调用;创作者在绘制草图时,可穿戴设备通过感知其动作及行为,联动外接知识库,触发跨界知识共享。未来脱离界面的协作,可以将知识共享从手动检索转化为情境化激发,使跨时空的知识碎片能在人类行为中即时拼接,为个体经验与集体智慧在动态交互中计算、共享与积累提供新路径。

However, the potential of such open knowledge sharing remains constrained by the core bottleneck of data privacy and security knowledge, as a fusion of structured data and tacit experience, cannot flourish if privacy concerns deter individuals and organizations from contributing\cite{jain2016big}. Resolving this tension requires multi-dimensional institutional and technological coordination\cite{wu2012effect}: Technologically, major software providers must implement dual-layered anonymization, permission, tiering safeguards for embedded databases, such as federated learning to enable cross-domain knowledge transfer without exposing raw data, or differential privacy to ensure shared data cannot be reverse-engineered to identify individuals\cite{jain2016big}. Institutionally, adaptive intellectual property frameworks are needed to clarify rights boundaries in cross-domain knowledge transfer, for example, whether bio-inspired analogies must comply with ethical norms from their source domains, or how user-generated analogical cases should be attributed and reused\cite{wu2012effect, grossman2004international}. Meanwhile, open-source communities play a pivotal role by contributing sanitized datasets (e.g., de-identified cross-industry service design cases)\cite{tamburri2019discovering} or high-quality Synthetic Data Generation(SDG)\cite{bauer2024comprehensive}, enriching shared resources while mitigating privacy risks. This ``technical safeguards + institutional norms + community co-creation'' paradigm aims to build trust in secure sharing, only when contributors are assured their data won’t be misused can cross-spatiotemporal knowledge flow evolve from fragmented exchanges to systematic accumulation.  
% 然而,这种开放式知识共享的潜力,仍受限于数据隐私与安全这一核心瓶颈——知识的本质是结构化数据与隐性经验的结合,而隐私保护的缺失会直接抑制个体与组织贡献知识的意愿。解决这一矛盾需要多维度的制度与技术协同:在技术层面,大型软件厂商需为嵌入工具的数据库构建“模糊处理—权限分级”的双重防护体系,例如通过联邦学习实现知识特征的跨域迁移而不泄露原始数据,或采用差分隐私技术确保共享数据无法反推个体信息;在制度层面,需完善知识产权保护的适配性规则,明确跨域知识迁移中的权利边界——如生物类比知识的使用是否需尊重其原生领域的伦理规范,用户生成的类比案例如何界定归属与复用权限。同时,开源社区的角色至关重要:通过贡献经过脱敏的通用数据集(如去标识化的跨行业服务设计案例),或基于真实数据生成高质量合成数据(如模拟不同文化语境下的用户反馈场景),既能丰富共享资源池,又能规避隐私风险。这种“技术防护+制度规范+社区共建”的模式,旨在构建“安全共享”的信任基石——唯有让知识贡献者确信其数据不会被滥用,跨时空协作中的知识流动才能从“碎片化交换”升级为“系统性积累”。


On a deeper level, the maturation of such multi-modal knowledge-sharing mechanisms will redefine the spatiotemporal dimensions of collaboration: Geographically dispersed teams could leverage dynamically updated analogical knowledge graphs to instantly access design experiences from diverse cultural contexts; cross-generational creators might rely on concept generation trajectories recorded by wearable devices, transforming tacit creative intuition into transmissible structured knowledge. The equilibrium between privacy protection and open sharing will ultimately ensure that knowledge accumulation neither becomes monopolized by a few nor loses value through chaotic circulation, this encapsulates the core challenge of future cross-spatiotemporal collaboration: empowering all participants to freely contribute and access knowledge within secure boundaries, allowing sparks of analogical thinking to collide across time and space, and ultimately coalescing into collective wisdom that drives innovation.
% 更进一步看,这种多形式知识共享机制的成熟,将重新定义“协作”的时空内涵:跨地域团队可通过动态更新的类比知识图谱,实时调用不同文化语境下的设计经验;跨代际创作者能依托可穿戴设备记录的“概念生成轨迹”,让隐性的创作直觉转化为可传承的结构化知识。而隐私保护与开放共享的平衡,最终将确保知识积累既不沦为少数主体的垄断资源,也不因无序流动而丧失价值——这正是未来跨时空协作的核心命题:让每个参与者在安全边界内自由贡献与获取知识,使类比思维的火花能跨越时空碰撞,最终沉淀为推动创新的集体智慧。


