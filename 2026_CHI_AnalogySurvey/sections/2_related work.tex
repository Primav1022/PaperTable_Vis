\section{Background}

\subsection{The Cognitive Mechanism of Design-by-Analogy}
%主要写analogy的认知机制,和design by analogy的将认知与实践紧密结合,与创造力的关系,与类比推理的区别,与case based reasoning的区别,以及其可计算的潜力。
Design-by-Analogy (DbA) refers to a design approach in which novel solutions within a target domain are created inspiration from a source domain via cross-domain analogical reasoning.\cite{jiang2022data, fu2014bio} This method centers around a design problem. By procedurally invoking external knowledge, experience, data, and other contents, it aims to stimulate operators' internal experience, memory, inspiration, and motivation. This method is a mechanism where cognition and practice proceed in parallel.\cite{moreno2015step, schecter2025role}.

Analogy, is a cognitive paradigm for the transfer of knowledge between domains\cite{gentner1983structure, hofstadter2001analogy} and it can be actively carried out when individual wants to acquire new knowledge\cite{gentner1997reasoning}. Design, on the other hand, has a strong practical attribute. As an intentional activity that aims to \textit{``transform existing conditions into preferred ones''} \cite{simon2019sciences}, design inherently operates as a goal-oriented practice, requiring physical or virtual implementation to validate conceptual solutions. Thus, Design-by-Analogy constitutes a goal-driven cognitive practice that naturally embeds the cognitive mechanisms of analogy within design processes. 
% 类比设计(DbA)是指一种设计方法,即通过跨领域类比推理从源领域获取灵感,从而在目标领域中创造出新的解决方案。\cite{jiang2022data, fu2014bio} 这个方法是围绕一个设计问题展开的,其通过程序化的调用外部知识、经验、数据等内容,致力于激发操作者的内部经验、记忆、灵感和驱动力。这种方法是一种认知和实践并行的机制。
%类比是一种用于在不同领域之间转移知识的认知范式\cite{gentner1983structure, hofstadter2001analogy},当个体想要获取新知识时,它可以被主动运用\cite{gentner1997reasoning}。另一方面,设计具有很强的实践性。作为一种旨在“将现有条件转变为理想条件”的有目的的活动\cite{simon2019sciences},设计本质上是一种以目标为导向的实践,需要通过物理或虚拟实现来验证概念性解决方案。因此,类比设计构成了一种目标驱动的认知实践,自然地将类比的认知机制融入设计过程中。 
%-------------------------【-DbA介绍-】---------------------

Design-by-Analogy demonstrates several key characteristics. First, the DbA process integrates the analogy-making process\cite{french2002computational} and the computation-able mechanisms\cite{gentner2011computational}, structured into four stages: representation/ encoding, retrieval, mapping and evaluation\cite{jiang2022data, fu2014bio, fu2015design}. Second, DbA usually has an \textit{``open goal''}\cite{tseng2008role} and typically combines intrinsic and extrinsic triggers\cite{moreno2015step, moreno2016overcoming}, enabling the human brain to link stimuli with memory concepts for the generation of ideas\cite{goucher2019neuroimaging}. Third, empirical and neuroinformatic evidence shows \cite{moreno2015step, goucher2019neuroimaging} that DbA can assist designers in overcoming design fixation through stimulation and analogical exploration.
%类比设计具有一系列特点,首先,类比设计过程综合了类比制作及计算机制,分为四个阶段表征/编码,检索,映射,评估四个阶段,通过相似特征的结构化映射将检索到的源域内容转换为目标域内容。其次,类比设计通常拥有一个开放性目标,结合内部刺激和外部刺激方法,使人脑将刺激与记忆概念连接,生成新想法。第三,类比设计被实证\cite{}和神经信息\cite{}手段证明可通过刺激和类比探索来辅助设计师突破设计固定。
%------------------------------【-DbA的特点-】--------------------


This practice-oriented nature enables DbA to directly harness analogical reasoning\cite{gentner1983structure, gentner2011computational, linsey2008increasing} for creativity support through three core mechanisms: First, DbA embeds the structured cognitive mechanisms\cite{french2002computational} into the creative process to scaffold the analogical reasoning of individuals in different stages of creative work\cite{hsueh2024counts}. Secondly, human goals and internal values will guide the direction of DbA divergence. In DbA, analogy is not aimless, but is constrained by human factors such as design goals, design factors, design problems, and the internal values of designers\cite{moreno2015step, chan2011benefits}.
Third, DbA internalizes the creativity computation. Due to the common analogical structure as \textbf{`` A:A'=B:B' ''}\cite{gentner1983structure}. The explicit cognitive operations in this process allow for algorithmic implementation and automated deployment, Methods such as symbolic models\cite{falkenhainer1989structure}, connectionist model\cite{thagard1990analog, hummel1997distributed} and hybrid models\cite{hofstadter1984copycat}, has been applied in many industrial scenarios. DbA organically integrates human cognition, creative behavior, and computer automation via the above three aspects, supporting creative computability without compromising human will.
%这种以实践为导向的特性使设计行为分析能够直接利用类比推理——检索和映射源 - 目标关系——通过三种核心机制来支持创造力:
% 首先,DbA 将类比构建的结构化认知机制(识别、细化、迁移、巩固\cite{french2002computational})融入创作过程,以支持个人在不同创作阶段的类比推理。【类比设计的基本认知实践相结合的能力】
% 其次,人的目标和内在价值会指导类比设计的发散方向, 在这之中,类比不是漫无目的,而是受设计目标,设计因素,设计问题,设计师的内在因素等人因约束。【类比设计的人因本质】
% 第三,类比设计将创造力计算内化。这一过程中的显式认知操作允许算法实现,例如符号模型\cite{falkenhainer1989structure}、联结主义模型\cite{thagard1990analog, hummel1997distributed}和混合模型\cite{hofstadter1984copycat},并且已经进行工业应用。【类比设计可自动化】
%类比设计通过以上三点将人的认知、人的创造行为、计算机自动化有机融合,支持创造力可计算的同时,不会丢弃人的意志。
%---------------【DbA激发创造力的机制】-------------------------


Unlike other creativity supporting methods, Design-by-Analogy have several advantages. In converting novelty to practicality, DbA outperforms than unstructured brainstorming\cite{linsey2010study}. When prioritizing the activation of human creativity, DbA reduces cognitive load compared to pure analogical reasoning\cite{richland2013reducing}. When compared with Case-Based Reasoning (CBR) method, for it's automated paradigm in retrieve-reuse-refine-restore, DbA involves a higher degree of human factors and creativity remaining\cite{wills1994towards} which can facilitate more on human intrinsic value stimulating.

% 与其他支持创造力的方法不同,在将新颖性转化为实用性方面,类比设计优于无结构的头脑风暴\cite{linsey2010study}。在考虑激发人类创造力方面,与纯粹的类比推理相比,这种综合方法降低了认知负荷\cite{richland2013reducing}。与基于案例的推理(CBR)相比,由于CBR在检索 - 重用 - 改进 - 恢复方面采用自动化范式,而DbA涉及更高程度的人为因素和创造力\cite{wills1994towards}。
%-------------------【-DbA和其它方法的对比-】---------------------------------









\subsection{Related Survey}

The concept of Design-by-Analogy develops after the birth of the structure mapping theory in 1983\cite{gentner1983structure}. Between 1983 and 2010, substantial foundational work emerged in this field, including evaluation metrics\cite{mcadams2002quantitative} and computational models\cite{french2002computational, gentner2011computational, falkenhainer1989structure, thagard1990analog, hummel1997distributed, hofstadter1984copycat}. Around 2008, after Linsey et al. introduced the WordTree method to Design-by-Analogy\cite{linsey2008increasing}, a series of studies applying this method to mechanical design\cite{fu2015design}, biological analogical design\cite{fu2014bio}, conceptual design\cite{moreno2016overcoming}, etc.


Today, DbA widely used in various fields of human activities. In biomimetic industry, classic cases in the wild include scientists inventing radar based on bat echolocation\cite{fu2014bio}. In academic, researchers have proposed function-behavior-structure design frameworks\cite{helms2009biologically}, related taxonomies\cite{fu2014bio}, and challenges\cite{nagel2018establishing, linsey2013overcoming}. In mechanical design, engineers developed more efficient wind turbine blades inspired by humpback whale fin tubercles which is already widely deployed in real world\cite{linsey2008increasing}. Studies in mechanical design field, focus on constructing patent migration databases\cite{fu2015design} and methods to prevent design fixation\cite{atilola2015representing, marshall2016analogy}. In product design, designers have drawn analogies from dump trucks to create cat litter boxes\cite{linsey2012design} and research efforts on analogy migration methods\cite{barnett2002and}, specific techniques\cite{ghane2024semantic,liu2023smfm, regenwetter2022deep, ghane2024semantic,}, design synthesis\cite{chakrabarti2011computer}, and interactive systems embedding cognitive mechanisms\cite{goel2012cognitive}. In education, scholars explored the analogy methods in education design\cite{duit1991role}. In philosophy and mathematics, DbA is studied as design metacognition\cite{ball2019advancing, holyoak1996mental} and a mathematical problem-solving and situation awareness approach\cite{polya2020mathematics}.

%%1983年结构映射理论诞生后,基于案例的类比推理随之发展起来\cite{gentner1983structure},在1983-2010年间,本领域有许多基础工作诞生,如相关的评价标准,计算模型等。在2008年前后,由linsey等人将wordtree方法引入类比设计领域后,一系列使用此方法的工作专注专利数据,生物数据,众包创意等应用层出不穷。至今,基于类比的设计行为广泛存在人类活动的各个领域中。在仿生工业领域,仿生设计中的代表案例是科学家依据蝙蝠的回声定位原理发明雷达,研究人员提出了功能-行为-结构的设计框架【】以及相关的分类法【】和挑战【】;机械设计中,工程师根据座头鲸的鲸鳍结节发明更高效的风力涡轮机叶片,研究人员专注于构建专利迁移数据库【】以及预防设计固定的方法【】【】;产品设计中,有设计师根据自卸卡车设计猫砂盆的先例,研究人员致力于研究各类方法,各类数据库,产品本体的类比迁移以及内嵌认知机制的交互研究【】【】【】;在教育行业中,研究人员使用类比设计【】【】【】研究教科书的源域近域等等;在哲学和数学领域,也有研究纯粹的类比作为设计元认知的,类比作为数学解题和情境感知法的内容【】【】
%--------------------------【-类比设计的历史和在各个领域中的应用及survey-】----------------


%我们对调研的相关调查分类并指出了相关的研究gap
We classified the relevant surveys and pointed out the research gaps. Surveys focusing on data include Jiang et al.'s study on data-driven analogical design theory \cite{jiang2022data}. Linsey et al.'s research on data representation and modality\cite{linsey2008modality}, and efficiency in DbA\cite{linsey2010study}. Barnett et al.'s focus on knowledge representation and transfer in analogy \cite{barnett2002and}. Fu et al. and Nagel et al.'s investigation into taxonomies of analogical design across different data types and themes \cite{fu2014bio, nagel2018establishing}. What’s more, O'Rourke et al.'s work on using BID data to address energy efficiency in mechanical design \cite{o2015toward}.
Surveys on algorithms and techniques include Chakrabarti et al.'s review of computer-based design synthesis \cite{chakrabarti2011computer}. Ghane et al.'s exploration of semantic Theory of Inventive Problem Solving (TRIZ) techniques in design \cite{ghane2024semantic}. Regenwetter et al.'s focus on deep generative models for engineering design \cite{regenwetter2022deep}. Dimassi et al. focused on the application of knowledge recommendation in 4D printing \cite{dimassi2023knowledge}. Liu et al.'s development of an analogical retrieval tool for innovative product concept design based on the structure-mapping function model (SMFM), proposing eight mapping types \cite{liu2023smfm}. Moreover, Aamodt et al.'s case-based reasoning (CBR), which has informed numerous computational models and techniques and shares similarities with DbA \cite{aamodt1994case}. Khanolkar et al.\cite{khanolkar2023mapping} mapped 7 AI methods to 5 design processes.
Surveys on interaction and evaluation include Tseng et al.'s study on the impact of analogical design provision timing \cite{tseng2008role}. Marshall and Atilola's research on analogy representation, proposing mechanisms affecting design fixation and creative processes \cite{atilola2015representing, marshall2016analogy}. Goel et al.'s exploration of cognitive science-empowered computer-aided design systems \cite{goel2012cognitive}. Verhaegen et al.'s review of existing idea evaluation metrics \cite{verhaegen2013refinements}. Lu et al.'s analysis of analogical design impacts across different stages \cite{lu2023differences}. In addition, Song et al.'s investigation into how different analogical information influences participants \cite{song2018characterizing}.
%专注于数据的survey包括Jiang等人专注数据驱动的类比设计理论\cite{jiang2022data},lindsey等人研究类比设计过程中数据的呈现,模态及效率【】【】,barnett等人关注类比中知识数据的表征和迁移\cite{barnett2002and},fu等人研究不同的数据和不同主题的类比设计分类法\cite{vio,patent},O等人则关注使用BID数据解决机械设计中的能源效率问题\cite{o2015toward},
%专注于算法和技术的文章有:Chakrabarti等人调研了基于计算机的设计合成\cite{chakrabarti2011computer},ghane等人探索了设计过程中使用语义Theory of Inventive Problem Solving(TRIZ)技术的可行性\cite{ghane2024semantic},Regenwetter等人专注深度生成模型如何应用于工程设计\cite{regenwetter2022deep},Dimassi 等人关注知识推荐在4d打印中的应用\cite{dimassi2023knowledge}。此外,liu等人基于structure-mapping function model (SMFM) 调查了创新产品概念设计类比检索工具并提出了八种映射类型\cite{liu2023smfm}。Aamodt等人提出的casebased reasoning(CBR)支撑了许多计算模型和技术的产生,与类比设计有异曲同工之妙\cite{aamodt1994case}, Khanolkar等人将7种ai方法映射在5个设计流程中。
%专注于交互和评估的survey包括Tseng等人关注不同时机提供类比设计的影响\cite{tseng},marshall和atilola等人关注类比表示研究,提出了影响设计固着和创造过程的不同方式\cite{atilola2015representing, marshall2016analogy},Goel等人探索了基于认知科学赋能的计算机辅助设计系统\cite{goel2012cognitive}, verhaegen等人调查了现有的想法评估指标\cite{verhaegen2013refinements},lu等人调查了不同阶段使用类比的不同影响\cite{lu2023differences},song等人调查了不同类比信息对参与者的影响\cite{song2018characterizing}

%--------------------------【-类比设计在数据、算法和技术、交互和评估三个方面的survey】----------------





Research gaps remain in the literature on DbA surveys: Firstly, existing reviews predominantly focus on specific disciplines or technical approaches, treating DbA as a tool serving fragmented domains, rather than considering disciplines and technical means from a panoramic perspective with human creativity itself as both the starting point and the objective. While a related survey\cite{jiang2022data} addresses mechanical design and data-driven approaches, it overlooks ambiguous data and applications beyond its scope. Researchs map the semantic TRIZ and AI techniques into creative stages but neglects designers' creativity supportive\cite{khanolkar2023mapping, ghane2024semantic}. Secondly, current work concentrates on single data sources or implicitly assumes DbA supports only on text, ``function-behavior-structure'' ontology and visual, ignoring other representations. Thirdly, disparate taxonomy persist across disciplines, lacking a unified framework centered on the creative process. This paper establishes a systematic review from the perspective of computational creativity, positioning human creative behavior as the core subject. By investigating the representation of DbA, summarizing techniques based on the creative process, and examining its application domains, we explore how DbA can augment individual experience-based creativity, bridge epistemological gaps, and reinforce human subjective value in the AI era.
%当前关于类比设计的调查有一些尚且没有被填补的空白:首先,当前综述多聚焦特定学科(如机械、生物、增材制造)或技术手段(如NLP-TRIZ、CV、3D打印),将类比视为服务分散领域的工具,而非以人类创意本身为出发点和目标全景式地考虑学科及技术手段。相近研究【】局限于机械设计与数据驱动,忽略模糊数据及其他领域应用,【】【】将技术方法映射在创造阶段中,但是忽略了设计师的创造力因素;其次,现有工作关注单一数据源(如专利数据、仿生学、岩土制造),或默认类比设计仅支持文字、功能或形态等模态数据,未考虑类比设计的其他表现形式;再者,各学科存在割裂的分类体系,缺乏创造过程视角为中心的统一分类框架。本文以计算创造力为研究视角,以人的创造行为为主体,建立系统性综述,通过调研类比设计的表现形式,总结基于创造过程的类比设计技术以及类比设计的应用领域,探索类比设计如何辅助个体基于经验的创造力,弥合认知论断层,在AI时代强化人的主体性价值。

%--------------------------------【目前研究gap】




























%-------------------draft---------------

%2. (1)目前没有根据设计创意阶段解构的类比设计survey,大多数survey关注的是用于模拟认知方法开发的技术、将类比设计应用于某一具体设计内容的,如机械,产品,生物,增材制造(AM)等,他们关注的视角本身就不是创造这件事情本身为中心,而是类比这种方法怎么服务于特定的分散的学科,与我们调研scope最相近的是这篇【】但是他的关注重点在机械设计和数据驱动,忽略了模糊数据和除机械设计以外的类比设计。(2)还有许多survey讲的是某一具体技术怎么进行类比的辅助,如nlp-triz技术,cv技术,3d打印技术怎么辅助类比这个工作而已,他们的聚焦中心是某一特定的技术领域,多类比设计中非该技术手段可以辅助的部分他们便不考虑了,也就是说他们不考虑类比设计的最终结果,而是考虑他们的技术能做哪一部分的类比,哪一个侧面的类比。(3)聚焦于某一特定数据来源/模态,比如patent data(专利数据),bio mimicry,岩土制造,机械,图形数据库等,单一文字或者单一图片的survey模态,他们考虑的是单一来源的数据,关注点是如何利用某一大型数据集来辅助特定的类比项目。
%【我们关注类比辅助的创意过程本身,是一个whole picture的survey,探索如何用这种方式更好的辅助人的创造行为】

%3.目前类比存在多元分类,在不同学科领域呈现出差异化的分类形式,尚未有统一的创意过程视角下的分类。在哲学领域,依据《哲学词典》的分类体系,类比可分为质料类比、形式类比、综合类比、对称类比、因果类比与协变类比六种形式。质料类比基于事物间的物质属性相似性进行分类,如将不同金属的导电性类比;形式类比则侧重于事物结构或关系的相似性等。这些分类主要依据类比所涉及的属性类型与关系特征进行划分。而在数学领域,类比主要包括降维类比、升维类比、结构类比等。在认知科学领域也是单独的定义类比,在机械设计中相关的survey也有许多,包括也有许多理论和技术在分布似的散点一样的在各大研究领域中展开,他们都为本工作提供了重大灵感支撑。【这一部分没人做过】

%% 4. 其他survey没有将类比设计当作一种创造力机制作为基础而将其他领域知识当作垂类知识库来进行系统review,没有将人的创造行为抬升到某一前所未有的高度,这一工作可以弥合许多领域的认知论gap,我们希望人关注到个体的认知行为的背后原因是希望人在ai时代通过有意识地觉察自己的思考逻辑和期待的事情发展方向,建立自己的主体性,关注人本身的特质,以及我们的愿景是希望individual可以通过计算到达的多元的发散的多彩的,而非统一的单一价值取向的目的地,探索一条新的渠道,释放不同个体的基于经验和个性化特征的创造力。本survey辅助的则是这一过程,不拘泥于任何强调非人创造行为以外的学科、技术、技巧、社会议题等干扰项。


















