\section{Findings}
This section outlines insights derived from the above content, addressing five relevant questions: technical guidelines, prospects for DbA applications in AI era, how DbA as a technology mediation to stimulate human intrinsic values, current research gaps, and their significance.
%这一章节我们描述从上述内容中获得的启发,回答相关的五个问题,包括技术指导,ai时代的类比应用前景,类比设计作为技术中介对人内在因素的刺激,目前的研究gap以及其意义。


\subsection{Why can DbA be used in a broader context? What is the significance behind it?}
%Q5:为什么要在更广泛的语境中使用类比设计?其背后的意义是什么?
To answer this question, we will review the difference between DbA method with other methods. Although Case-Based Reasoning (CBR) as a mainstream methodology widely adopted across industries for its near-autonomous automation with standardized efficiency, its critical flaw lies in the almost exclusion of human volition, which constraining the integration of emergent culture and rendering empirical legacy systems inadequate for contemporary demands, thereby perpetuating constraints and oppression\cite{wills1994towards, goel2012cognitive}. Conversely, Design-by-Analogy (DbA) organically executes CBR processes while uniquely balancing new scenarios\cite{Ju2025toward}, cultural contexts\cite{cao2025medai}, and designer intent with intrinsic experiential knowledge stimulate mechanism\cite{warner2023interactive}. As evidenced by empirical study, when interactive human selection of analogical mapping logic, the behavior itself become a bridge to connect the computational workflows between the source domain and the target domain, the trajectories of a certain behavior can steer design outcomes toward divergent, more personal and more creative outcome\cite{chan2011benefits, moreno2015step, moreno2016overcoming}. This approach inherently challenges industrial efficiency dogma, for its embedded structured mapping logic not only seamlessly integrates with existing industrial processes and manufacturing methodologies but also instills an indispensable creativity-centric foundation. As industry values shift from efficiency toward personalization and customization, DbA’s judicious application in manufacturing holds potential to realize mass customization and the Manufacturing-as-a-Service (MaaS) vision\cite{zhong2017intelligent}. Furthermore, by strategically leveraging the emergent capabilities of large language models(LLMs), DbA can provide domain-specific recommendations alongside novel perspectives, concepts, and cultural insights during analogical retrieval, categorized by mapping distance, while at the mapping stage, it delivers unprecedented logical frameworks distinct from prior AI methods.
% 1. 当下许多产业使用的是CBR case based reasoning,这一方法几乎可以做到无人自主的全自动化流程,在其标准化、高效的优点之中,存在几乎不涉及到人的意志的缺陷,这导致新型的前沿的理念无法嵌入的同时,基于经验主义的旧有方法不能完全适配新时代,形成了拘束与压迫。而DbA具有可以有机执行cbr流程,但是基于新的场景,文化,设计师意图,同时平衡内在经验的潜力. 实证研究表明,当人类对类比映射逻辑进行交互式选择时,这种行为本身就成为连接源领域与目标领域计算工作流程的桥梁,特定行为的轨迹能够引导设计结果朝着更具发散性、个性化和创造性的方向发展\cite{chan2011benefits, moreno2015step, moreno2016overcoming}。
% 2. 本方法可以挑战工业产业的效率至上主义,因其内嵌的结构化映射逻辑可以在完全承接当下工业产业流程规划及制造方法迁移的同时,为工业产业赋予难以被摒弃的创造力基底。在以效率为导向的工业价值观逐渐转向为以满足个性化为导向的时代,dba在制造业中的合理应用有潜力实现大规模个性化定制以及MaaS的愿景。
% 3. 通过合理的利用大模型的涌现能力,可以在类比启发中根据映射距离的长短,通过检索提供专业建议的同时提供新的观点,理念,文化等人文因素;在映射阶段可以提供新颖的逻辑,区别于之前的人工智能方法。

%大模型能力增强以后可以理解更加深度的关系,讲一下具体是什么新的场景,除了工业制造的,引用下https://ppyyqq.github.io/aicc/
%--------------整合------------


\subsection{What common guidance can be provided for DbA techniques?}
%Q1:有哪些技术上可以提供的共性指导?
We elaborate on the technical guidance discovered in the article and summarized by ourselves from the aspects of \textbf{user intention}, \textbf{data}, \textbf{algorithms}, and \textbf{interaction design}.
%我们根据用户意图、数据、算法、交互设计几个层次来阐述我们在文章中发现的以及自己总结的技术指导。

%Intention awareness
\textbf{At the user intention awareness and discriminate}, we posit: 
1. The theoretical research findings on user behavior should be applied in the system.\cite{goel2012cognitive}. During the formative research phase of system design, cognitive experiments on the create behavior procedure transaction additional, such as EEG experiments documenting transitions in the creative process and corresponding behavioral mappings (e.g., pauses, eye movements, frequent clicks, browsing irrelevant websites), should be established to indicate when state transitions require inspirational assistance. Clearer mappings between human cognition, behavior, and machine-assisted actions should be considered and established\cite{borgianni2020forms, goucher2019neuroimaging}. Given the extended nature of creative processes, the development of combinatorial methods and algorithms for creative details is imperative. Based on intention recognition, user states should be assessed to switch analogy techniques, providing intelligent and timely external stimuli.  

2. Provide far-field inspiration and near-field operation inspiration to experts with a relatively deep knowledge level. Empirical studies confirm that experts' thinking differs from novices'\cite{chai2015behavioral, chen2024toward}, involving distinct brain regions and cognitive structures\cite{goucher2019neuroimaging}. Thus, after comprehensively analyzing and awarnessing the needs of experts and novices at different create stages, efforts should focus on developing universally useful tools with differentiated support. For instance, novices in the initial creative phase should receive \textit{close-field structured resources}(Function-Behavior-Structure) to lower cognitive barriers, while experts should be provided with \textit{far-field stimuli} and metacognitive tools\cite{ball2019advancing, viswanathan2016study} early on to activate intrinsic experience, followed by specific near-field stimuli in later stages to refine designs.
% 在用户意图理解,我们认为:
% 1. 将对用户行为的理论研究结果在系统中加以应用。在系统设计的前期形成性研究阶段,设立更多认知实验,如脑电实验记录创造过程的切换和识别以及对应的行为映射,如停顿、眼动、频繁点击、浏览无关网站等行为预示着状态的切换需要灵感的辅助等。将人的认知、行为和机器辅助行为之间建立更明确的联系和映射关系。因为创造过程的拉长,组合方法的开发和创作细节上的算法开发势在必行。根据意图识别,判断用户的状态来切换类比技术的使用,提供智能的适时的外部刺激。
% 2. 对具有较深知识水平的专家提供远域灵感启发,提供近域操作启发。对知识深度不同的专家和新手制定分开的类比设计策略。【towards2024chen】实证研究证明,专家的思维和新手不一样,调用的脑区和思考结构不同。所以要在全面思辨专家和新手不同阶段的需求之后,致力于开发给专家和新手同时有用的东西,差异化辅助,如,针对新手第一阶段的创造行为,应提供近距离结构化资源(功能-行为-结构)降低认知门槛,而针对专家则应在一开始提供远场刺激,辅助元认知工具,以激发其内在经验,而在后期提供特异性的近场刺激,精细打磨设计。

%先用gpt过一遍中文


\textbf{At the data processing level}, we contend: 
1. Techniques and methods for encoding experience into structured data with universal relevance. Expanding the understanding of ``structured data'' to the meta-logic governing the operation of all things, including abstract information such as experience, processes, causality, and phenomena\cite{bandemer2012fuzzy}. A prime example is the inherently structured relationship of function-behavior-structure\cite{chen2024toward}, with similar logic extending to frameworks like Input-Processing-Output\cite{boell2012conceptualizing}. Such ontologically structured databases lay the foundation for vertical specific knowledge assistance in future agentic AI\cite{acharya2025agentic}.  

2. Reducing the scenario dependency of data applications and enhancing data reuse efficiency and cross-scenario adaptability. For example, systematic Embedding at Scale: Avoid fragmented single-point data by constructing domain or logic based data module (e.g., a ``manufacturing process data module'' encompassing sub-modules like equipment parameters, operational workflows, and fault resolution), forming systematically batch-callable data pools. Transferable Loading Methods: Enable rapid data adaptation across systems through standardized interfaces, universal mapping rules, or middleware protocols. For instance, designing ``manufacturing process experiential data'' as transferable modules allows embedding into Manufacturing-as-a-Service(MaaS)\cite{zhong2017intelligent} systems or AI chatbots without redundant development. 

3. Expert led production of vertical specific knowledge bases/experience repositories/prompt libraries/analogy mapping rules is essential\cite{gentner1983structure, moreno2016overcoming}. The key lies in establishing and iterating cross-domain mapping logic, interdisciplinary analogy lexicons, and analogy relationship tables\cite{cao2025medai} based on professional practice addressing core domain pain points to unleash the native creativity of domain knowledge.
% 在数据设计层次,我们认为:
% 1. 将经验收录为具有普适相关性的结构化数据的技术和方法。将对“数据”的固有认知扩展到万事万物运行的元逻辑上,不仅是传统的数值、文本等内容,经验、流程、因果关系、现象等抽象信息均可视为数据。一个很好的了例子就是function-behavior-structure的天然结构化联系,相似的逻辑还包括Input-Processing-Output等,这一类具有结构化联系的本体数据库可为为未来的agentic ai打好垂类知识辅助的地基。
% 2. 降低数据应用的“场景依赖”,提高数据的复用效率与跨场景适配,包括:成规模成体系的嵌入:避免单点数据的碎片化,而是按领域或逻辑构建“数据模块集群”(如“制造业工艺数据模块”包含设备参数、操作流程、故障处理等子模块),形成可批量调用的体系化数据池。可迁移的载入方法:通过标准化接口、通用映射规则或中间层协议,让数据能快速适配不同系统(如API接口、通用数据格式、预训练模型的参数迁移)。例如,将“制造工艺流程经验化数据”设计为可迁移模块,既能嵌入制造即服务系统,也能适配智能问答机器人,无需重复开发。
% 3. 专家主导垂类的知识库/经验库/提示词/类比映射规则的生产,关键在于基于专业实践经验建立和迭代跨领域的映射逻辑,跨学科的类比词汇表,类比关系表等等,是最能覆盖领域核心痛点的,进而才能释放领域知识的“原生创造力”。

\textbf{At the algorithm design level}, we argue that: 
1. Algorithm design that balances automation and human participation. When deploying a certain algorithm, designers should assessment the specific context and the goal of leverage DbA for problem solving, for instance, in DbA methods as we collected, existing studies have shown that the\textit{\textbf{Wordtree}} method is effective in alleviating design fixation but imposes high cognitive load because it's amount imformation, while the \textit{\textbf{SCAMPER}} method generates more novel ideas through systematic method guidance yet has limited effect in overcoming fixation\cite{moreno2016overcoming}. The \textbf{\textit{TRIZ}} method, which tends to use logical thinking and tools to solve inventive problems\cite{ilevbare2013review}, is unsuitable for non-logical issues, and the \textbf{\textit{CBR}} method only applies to scenarios with existing cases, not unknown contexts\cite{wills1994towards}. Rule-based symbolic models work for simple execution scenarios with high response velocity, while connectionist models based on edges, nodes, weights, and networks are applicable to contexts trained on structured data but not complex problems\cite{jiang2022data}.
% 在算法设计层次,我们认为:
% 1. 平衡自动化和人参与的算法设计。不同的方法适应不同的情境,比如类比设计方法中,已有研究证明“wordtree方法适用于缓解设计固定但是认知负荷高,scamper方法通过系统性问题引导生成更多新颖想法,但在克服固着方面效果不明显”;TRIZ方法,倾向于使用逻辑化的思路和工具来解决发明问题,不适用于非逻辑的问题, cbr方法仅适用于有案例的情况,不适用于未知的情境;基于规则的符号模型适用于简单执行场景,基于边、节点、权重和网络的连接注意模型适用于执行基于结构化数据训练的情境,不适用于具有复杂性的问题等。

%第一句话别太笼统,

2. Long distance analogy is suitable for scope broaden, while short distance is suitable for exploration deepen, with variations in methods and tendencies of long/short-distance mapping across different design phases: during the \textit{\textbf{Vision}} phase, a combination of long and short-distance mapping is suitable; the \textit{\textbf{Inspiration}} phase calls for long-distance analogies\cite{kang2025biospark}, whereas the \textbf{\textit{Ideation}} and \textbf{\textit{Prototype}} phases tend to gradually converge, for which we propose a gradual reduction in the number of long-distance analogies, with the prototype phase requiring short-distance analogies to provide lateral support with detailed knowledge\cite{tseng2008role}; in the \textbf{\textit{Fabrication}} phase, when designers are unsure how to proceed, they should first conduct analogical retrieval of content from other domains based on similar concepts\cite{emerson2024anther, schulz2014design}, then use short-distance analogies once processes or workflows are determined\cite{dimassi2023knowledge, khosravani2022intelligent}; and the \textbf{\textit{Evaluation}} phase involves case-based collaborative long and short-distance analogies\cite{thomas2013extending}. Additionally, the required distance of analogies varies with needs, for example, a focus on quantity leans toward short-distance and uncommon examples, while a focus on quality demands cross-domain long-distance mapping—and with users, as novice designers need near-field suggestions, experts require more far-field advice, and different approaches to providing analogical inspiration are needed for people with different personalities due to their varying internal experiences\cite{chan2011benefits, song2018characterizing, chai2015behavioral}. 
% 2. 类比长距离和短距离启发也应该分情境应用,不同设计阶段的长短距离映射的方法和倾向是不同的,如在vision阶段,适用于远近距离映射结合;在inspiration阶段适用远距离类比,而ideation阶段和prototype阶段则倾向于逐渐收敛,我们认为应当逐渐减少远距离类比出现的数量,在prototype阶段则需要近距离的类比,横向提供细节知识的辅助;在fabrication阶段,在设计师不知道如何开展的时候应该先进行基于同类概念的其他领域内容的类比检索,确定工艺或者流程之后进行近距离的类比;在evalution阶段,提供基于案例的远近协同类比。在需求不同时,需要的长短类比距离也不同,如,需要数量的时候倾向于近距离实例以及不常见实例,在期待质量的时候则需要跨领域远距离映射。在面对用户不同时也不同,设计新手需要近场建议,专家需要更多远场建议,给不同性格的人(因为他们的内部经验不同)也有不同的类比灵感提供方法。
%类比长距离适合广度任务,短距离适合深度探索

3. Retrieval methods, including those based on FBS ontology\cite{chen2024toward}, knowledge graphs\cite{huang2024field}, cases\cite{lupiani2017monitoring}, and streaming data\cite{gonzalez2018energy} and have diverse representation, with different research domains requiring distinct methods due to variations in knowledge structures, for example, biological analogies mainly use FBS\cite{chen2024BIDTrain, chen2024toward}, while process planning is case central\cite{zhang2020deep}. 
% 3. 检索的方式,有基于fbs本体的,有知识图谱base的,有case based,有流数据等,主要是数据驱动,表现形式多样。因为知识结构的差异,不同的研究领域应使用不同的方法,生物类比主要使用fbs,流程规划主要以案例为中心。

4. Mapping methods are more complex: their representation, as described in Sec. 4, include similarities from semantics to appearance, structure, function and etc., requiring distinct similarity calculation methods to map context, which means algorithms must provide accurate and appropriate representational mappings based on user identity, design phase, and application scenario. Current calculation methods and formulas, in addition to basic analogy formulas, include many innovative similarity calculation methods within the domain\cite{dimassi2023knowledge, hill2019learning, boteanu2015solving, marquer2024solving}, with future challenges lying in integrating user needs to provide multi-modal representations, achieving real-time dynamic representational calculation and conversion algorithmically, and, given cultural differences across domains, requiring domain experts to establish their own mapping rules for specific mass-produced workflows\cite{cao2025medai, dimassi2023knowledge}. The significance of DbA mapping lies in its role as a key distinction between this design method and approaches like case-based reasoning and analogy reasoning, as it is not a fully automated, globally managed method but requires integration of human designers' intentions at all stages, and integrating designers' perceptions, capabilities, experiences, and domain-specific advice is therefore equally crucial in algorithm implementation.
% 4. 在映射方法上,则更为复杂。这类规则的表现形式如Sec4所述,包括语义到外观到结构和功能相似性,不同的表征有着不同的相似性计算法方法,这需要算法基于用户身份、设计阶段、应用场景等,提供准确且适合的表征映射。目前计算方法与公式除了基本的类比公式以外,在领域内有许多创新的相似性计算方法\cite,难点在未来如何结合用户需要的内容提供各类模态的表征,且在算法上实现实时动态的表征计算和转换。其次,不同的领域因为文化差异,如果要提供特定的批量化流程,则需要领域专家自行建立映射规则。mapping的重要性在于,这是本设计方法区别于case based reasoning,analogy reasoning等方法的重要部分,因其并不是全局管理的自动化方法,在各个阶段都需要融合人类设计师意图,所以深度融合设计师感知、能力、经验、领域独特建议等在算法实现层次也相当具有重要性。

\textbf{At the interaction design level}, we assert that DbA tools diverge from traditional toolbox-like interfaces by not relying on foundational graphical logic (e.g., points/lines/planes) as core methodology; instead, utilizing visualizations such as charts\cite{emerson2024anther}, graphs\cite{yan2023xcreation, huang2024field}, tree diagrams\cite{chen2024BIDTrain, chen2024asknaturenet}, canvas\cite{lin2025inkspire, masson2025textoshop}, radial maps, double-bubble maps, structured mind maps, chunking and recombination action and etc. based on metacognitive forms\cite{ball2019advancing}, can explicitly reveal structured analogical mappings logic\cite{gentner1983structure}. 
2. Process collaboration necessitates seamless transitions, integrating text-image heuristics and human cognition while focusing on stage-specific demands: extensive information intake during early phases, meticulous refinement in mid-stages, and dashboard-driven visualization in later phases\cite{kang2025biospark, srinivasan2024improving}. 
3. Novel UI, dynamic effects, and experience design is needed, exemplified by distance-based UI/iconography, implicit interactions\cite{ju2008design}, translucency effects, and ephemeral appearances, serve to implicitly inspire analogical mappings direction, provide subtle cues, and facilitate connection establishment, avoid the design fixation and standardization of creativity caused by direct guidance.
% 在交互界面设计层次,我们认为:
% 1. 类比设计工具的交互界面区别于传统的工具箱似的界面,并非是从点线面等图形学底层逻辑构成内容,或者说他们并不是此类系统的核心方法。在我们语料库中的工作很多使用各类交互启发式界面设计,使用可视化图表、图、树状图、辐射图、双泡泡地图、结构化思维导图等基于元认知形式的界面设计来辅助类比设计,以清晰展示类比的结构化映射。
% 2. 流程协同,过程过渡应自然,结合图文启发、人类认知等,关注创造工作各个阶段的主要内容,如前期需求大量信息,中期需要精细打磨,后期则需要图表看板。
% 3. 新型的ui、动效、体验设计,比如远近距离等方法的ui和icon设计,使用隐式交互,半透明,忽隐忽现等等方式进行启发和类比映射启发,隐式提示,帮助建立映射等等。避免直接引导所带来的设计固定和创造力标准化。


\subsection{Where lies the potential of DbA application in AI era? What form does it take?}
%Q2:AI时代类比设计应用的潜力在哪里?是何种形式的?

At the pivotal moment when foundation models have fully entered the mainstream, dictated by societal divisions of labor\cite{hsueh2024counts}—exhibit distinct forms tailored to specific creative stages, with problem-solving approaches and real-world impacts varying accordingly\cite{shao2025future}; these domains encompass diverse interfaces such as canvas-based design tools, mobile management applications, data dashboards, web platforms, fine-tuned models for micro-manufacturing robots\cite{rigaud2022exploring}, MaaS-enabled intelligent service design software for industrial machinery\cite{zhang2020deep}, context-aware visual data projection systems integrated with projectors and etc. This fluid, pluralistic, and diverse formal adaptability, which is a hallmark of DbA methodology, exists expressly to fulfill creative works demands: tools for the \textbf{\textit{Vision}} phase manifest as market/user research-driven commercial planning instruments or cloud-based lightweight terminals/investment management systems; the \textit{\textbf{Ideation}} stage employs creative support platforms akin to \textit{Figma}\cite{figma2025}; \textbf{\textit{Prototype}} leverages CAD-style platforms for content interaction and migration; \textbf{\textit{Fabrication}} utilizes manufacturing simulation plugins to transfer domain-specific process parameters (e.g., humidity to extrusion rate mappings) enabling deployment of MaaS solutions; \textbf{\textit{Evaluation}} constructs data mart based operational dashboards incorporating supply chain variables for product lifecycle forecasting; \textit{\textbf{Meta}} phase with agentic reasoning\cite{wu2025agentic} powered intelligent scheduling systems that automate project resource allocation through cross-tool knowledge graphs.
%在基座模型全面进入主流的关键节点,受社会分工影响,各个领域的创造性产出呈现出为特定创作阶段量身定制的独特形式,其解决问题的方式和对现实世界的影响也各有不同;这些领域涵盖了多种不同的交互界面,诸如基于画布的设计工具(如CAD)、移动管理应用程序、数据仪表板、网络平台、用于微制造机器人的微调模型、适用于工业机械的具备MaaS功能的智能服务设计软件,以及与投影仪集成的情境感知视觉数据投影系统。这种灵活多变、多元且丰富的形式适应性——类比设计方法的显著特征——明确是为满足创造性需求而存在的:在愿景阶段的工具体现为以市场/用户研究为导向的商业规划工具,或是基于云的轻量级终端/投资管理系统;构思阶段采用类似Figma的创意支持平台;原型制作阶段借助CAD风格的环境进行内容交互与迁移;制造阶段利用制造模拟插件来传递特定领域的工艺参数(例如湿度与挤出速率的映射关系),从而能够即刻部署MaaS解决方案;评估阶段构建基于数据集市的运营仪表板,纳入供应链变量以预测产品生命周期;最终在元阶段,依靠自主推理驱动的智能调度系统,通过跨工具知识图谱实现项目资源的自动分配。

%分成大类,大类里面没涉及到的可以写一下,现有的potential,除此以外还有别的大类under explore,分为两个层,不用按照阶段来写了。

\subsection{What are the research gaps that have not been explored in academia currently? }
%Q4:目前学术界尚未开发的研究gap是哪些?
1. In the field of service design, there are a few applied studies using DbA methods, possibly due to the lack of data infrastructure in this domain, where few scholars have paid attention to it or even established reasonable experience transfer application designs, given the difficulties in empirical research and theoretical construction\cite{moreno2014analogies}. 
2. In intelligent manufacturing, there are few data collection methods and great challenges in experience appropriation, necessitating the development of reasonable and universal fuzzy data collection paradigms\cite{emerson2024anther, ross2022exploring}; furthermore, researchers in this field generally focus on DbA in the conceptual phase rarely transferring and applying DbA paradigms in the manufacturing phase, and thus the fixed perception that the manufacturing industry only has tool value should be changed to explore its potential for creative work\cite{hsueh2024counts}. 
3. Interaction forms remain relatively singular with few innovative ones, concentrating on software, while more forms, such as projection, augmented reality, virtual reality, brain-computer interfaces, and wearable devices, await exploration to inspire and stimulate DbA and assist production. 
4. The outputs of DbA need more empirical evaluations of their effectiveness, innovation, and usability, which are crucial for deploying such work in the real world\cite{verhaegen2013refinements}. 
5. Structured design knowledge bases and experience repositories are still lacking, including emerging design concepts like sustainability and circular economy required for general products\cite{liao2021priming}.
6. More fields need to explore the applications of DbA, such as environmental design\cite{ross2022exploring}, sound design, exhibition design, etc.
% 1. 服务设计领域使用类比迁移方法的应用研究较少,可能因为该领域数据基建的缺乏,鲜有学者关注到这一领域乃至建立合理的经验迁移应用设计,因实证研究难度大,理论建立难度大。
% 2. 智能制造的数据收集方法少,经验挪用难度大,需开发合理且通用的模糊数据采集范式。其次,本领域的研究人员关注的dba普遍集中在构思阶段,鲜少在制造阶段迁移和应用dba范式,应当转变对制造行业仅仅具有工具价值的这一固定认知,挖掘智能制造为创意工作的潜力。
% 3. 交互形式较为单一,创新交互形式较少,集中在平面软件,更多的交互形式有待探索。如投影,增强现实,虚拟现实,脑机接口、可穿戴设备等来启发和刺激类比设计,辅助生产制造。
% 4. 类比设计产出的内容缺少对其效能、创新程度、可用性等的实证测评,这对在真实世界部署这类工作十分重要。
% 5. 结构化的设计学知识库及经验库尚且缺乏,例如包括通用产品所需要的可持续、循环经济等新兴设计理念。
% 6. 更多的领域需要探索类比设计的方法,如环境设计和声音设计,展览设计等。



%可以写成一段话,大的设计学领域的其他一些问题




\subsection{How to understand DBA as a technology mediation that can stimulate the intrinsic value of human beings, and how does it deploy in real world?}
%Q3:如何理解dba作为一种可以激发人类内在价值的技术中介,在真实世界如何部署?

% 1. 技术中介是什么,设计哲学领域引用一些说明技术道德和伦理的段落
% 2. 类比设计为何是一种技术中介,结合技术手段维护道德伦理的本质
% 3. 为什么可以激发人的内在价值,塑造人的行为,以创造为中心,激活人的主体性
% 4. 在真实世界首先需要弥补许多设计师的这一认知论上的不足,在可以解放劳动力的时代探索解放的具体方法,同时保有对人的深切关怀,在新的时代,每一种技术开发出来都应该经过价值审查。
% 5. 然后他们根据所在的相关的行业有机的调用这些人本的设计方法,鼓励创新和创造,检测实际的效能,优化传统模式转型等等,才能慢慢转变,实现各行各业的更高的目标和理想。


Within design philosophy, Peter-Paul Verbeek’s theory of technological mediation transcends the simplistic view of technology as a neutral tool, positing instead that it actively shapes human-world interactions while embedding moral and ethical dimensions in its functionality\cite{verbeek2006materializing}. As Verbeek argues, technology is not merely an instrumental means but a co-constitutive mediator of human experience and behavior, influencing not only what we do but how we perceive value\cite{verbeek2011moralizing}.
% 在设计哲学领域,彼得-保罗·费贝克等学者所阐述的技术中介理论,超越了将技术视为中性工具的简单认知;相反,该理论认为,技术会主动塑造人类与世界的互动方式,并在其功能中嵌入道德与伦理维度。费贝克提出,“技术并非仅是达成目的的手段,更是共同构建人类经验与行为的中介”,这一观点表明了技术制品不仅影响“我们做什么”,更会影响“我们如何感知价值”。

This work contends that DbA operates as a form of technological mediation through systematic knowledge mapping, cross-domain transfer, and contextual adaptation, mechanisms that render it not a passive tool but an active shaper of design cognition. First, DbA guides designers’ reasoning while preserving their agency, as their intrinsic knowledge directs analogical trajectories in a dynamic interplay\cite{chan2011benefits}. Second, AI-driven DbA implicitly embeds ethical boundaries by prioritizing morally compliant knowledge transfers and filtering harmful analogies, thereby sustaining ethical integrity. Its mediation thus extends beyond fostering creativity to steering it toward responsible innovation\cite{liao2021priming, smits2019values}.
% 本文认为类比设计是一种技术中介,通过系统性知识映射、跨领域迁移与情境适配发挥作用——这些机制使DbA不仅是被动工具,更成为设计认知的主动塑造者。首先,类比设计可以引导设计师的推理过程,并因设计师独特的内在知识引导类比方向,在这样交互动态的过程中维护设计师的主导意志。其次,基于人工智能的类比设计可以隐含地植入伦理边界:其技术逻辑可以支持优先选择与道德规范相符的可迁移知识,过滤有害类比,从而维持伦理完整性。在此意义上,其中介作用不仅在于促进创造力,更在于引导创造力走向负责任的创新。

This mediating function enables analogical design to activate human intrinsic values, shape behaviors, and with creation as its core, provide a vehicle for humans to enhance their subjectivity\cite{schecter2025role}. By bridging disparate domains, DbA disrupts rigid cognitive patterns, compelling designers to actively construct meaning rather than passively execute predefined steps, a process that reinforces their role as subjects rather than objects of design\cite{schecter2025role, shao2025future, hsueh2024counts}. For instance, when drawing analogies between biology and machines, designers exercise judgment, curiosity, and ethical discernment, which in turn fosters holistic, human-centered solutions. Centering creation, DbA cultivates ownership and amplifies intrinsic motivations like mastery and purpose, key drivers of ethical and innovative practice. 
% 这种中介功能使类比设计能够激活人类内在价值、塑造行为,并以创造为核心,为人类提升内在价值提供载体。通过连接不同领域,类比设计打破僵化的认知模式,促使设计师主动参与意义建构,而非被动执行预设步骤——这种创造性整合的行为,强化了他们作为设计过程“主体”而非“客体”的角色。例如,当设计师在生态系统与城市规划之间建立类比时,他们并非仅是应用工具,更是在运用判断力、好奇心与伦理洞察力,而这反过来又会引导其行为转向更具整体性、以人为本的解决方案。通过以创造为核心,类比设计培育了人们对成果的归属感,放大了诸如精通感与使命感等内在动机,而这些动机正是伦理与创新实践的驱动力。

Yet, this potential remains constrained by epistemic limitations among practitioners who still perceive technology as neutral rather than mediation\cite{muukkonen2005technology}. Such oversight risks reducing DbA to a mechanical process, undermining its capacity to honor human agency. Moreover, in an era where automation liberates labor from routine tasks\cite{hsueh2024counts, shao2025future, zhong2017intelligent}, practitioners must explore how such ``liberation'' reorients work toward creativity, ethics, and social meaning, ensuring it empowers rather than displaces. This demands rigorous value assessments of emerging technologies, preempting unintended constraints on autonomy or perpetuated harm\cite{de2018can, smits2019values}.
% 然而在真实世界实践中,这种潜力仍受限于设计师群体中的认识论——许多设计师仍将技术视为中性工具,而非中介力量。首先,这种认知疏漏可能将类比设计简化为机械过程,削弱其尊重人类主体性的能力。其次,在技术日益自动化常规任务、将劳动力从枯燥工作中解放出来的时代,在野实践的从业者去探索这种“解放”的具体方法至关重要:这不仅是替代人力,更是将人力重新导向创造性、伦理性与社会意义的工作。这要求从业者深切关怀人类尊严——确保“解放”不等同于“剥夺权力”,并对每一项技术发展进行严格的价值审核,在部署前评估类比设计等工具可能如何不经意地限制自主性或延续伤害。

The work discussed herein gradually fills these gaps. By establishing DbA in the recognition of technology as a mediator of creation and ethics, many fields of research enable fields from education to manufacturing to transcend incrementalism\cite{johnson1988rethinking} and advance toward higher-order ideals of responsibility, empowerment and collective meaning.
% 本文收集的工作都在逐渐填补这些空白。这种转变植根于对类比设计是“技术是创造与伦理双重中介”的认知,能让从教育到制造业的各个领域超越渐进式改进,迈向负责任、赋权性与集体意义的更高设计理想。