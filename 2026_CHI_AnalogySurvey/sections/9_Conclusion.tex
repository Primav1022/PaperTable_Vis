\section{Conclusion}
This systematic review examines the cross-domain application of Design by Analogy (DbA) methodology driven by artificial intelligence, underscoring its pivotal role as a universal cognitive foundation in activating innate human capabilities and bridging fragmented research. The study demonstrates that DbA, through six representational forms (Semantics and Text, Visual and Appearance, Material and Structure, Function and Attribute, Interaction and Multi-sensory Experience, Unconventional Contexts), supports four phases of the creative process (Problem Definition, Product Ideation, Product Implementation, Product Evaluation) and enables innovation across creative industries, smart manufacturing, education, and services. Addressing three critical challenges—insufficient articulation of DbA mechanisms, limitations in human-AI collaboration technologies, and the absence of cross-domain application frameworks—this review proposes a theory-technology-application framework rooted in human internal experience. Thematic analysis of 1,615 screened publications and in-depth synthesis of 86 core articles establish six classification criteria, offering future research pathways for technological mediation and ethical value alignment. By validating DbA’s potential in cross-modal knowledge transfer and computational deployment while critically countering generative AI’s oversimplification of design processes, this work lays the groundwork for structured design repositories, fuzzy data acquisition paradigms, and novel interaction modalities.
%本文系统综述了类比设计(DbA)方法论在人工智能驱动下的跨领域应用,揭示了其作为共性认知基础在激活人类固有能力、弥合碎片化研究中的关键价值。研究表明,DbA通过六类表征形式(语义文本、视觉外观、材料结构、功能属性、交互与多感官体验、非常规情境)支持创意过程的四大阶段(问题定义、产品构思、产品实现、产品评估),并赋能创意产业、智能制造、教育服务等领域的创新实践。针对当前研究存在的三大挑战——DbA机制阐释不足、人机协作技术局限、跨领域应用体系缺失,本文提出以人类内在经验为根基的理论-技术-应用框架,结合主题分析确立的六大分类标准,为未来研究提供了技术中介化路径与伦理价值导向。基于对1,615篇文献的系统筛选与86篇核心文献的深度分析,本综述不仅验证了DbA在跨模态知识迁移与计算化部署中的潜力,更批判性回应了生成式AI对设计过程的过度简化,为构建结构化设计知识库、开发模糊数据采集范式及探索新型交互形式奠定基础。